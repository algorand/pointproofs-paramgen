\documentclass{article}
\usepackage{amsmath,amsthm,amssymb}

\newcommand{\G}{\mathbb{G}}

\newcommand{\gOne}[1]{g_1^{\alpha^{#1}}}
\newcommand{\gTwo}[1]{g_2^{\alpha^{#1}}}

\newcommand{\?}{\stackrel{?}{=}}

\newtheorem{claim}{Claim}

\begin{document}

\subsection*{Overview}

Public parameters are of the form
\begin{align*}
g_1^\alpha, \gOne{2}, \dots, \gOne{N}&,  \gOne{N+2}, \gOne{N+3}, \dots, \gOne{2N} \\
g_2^{\alpha}, \gTwo{2}, \dots, \gTwo{N}&,  \gTwo{N+2}, \gTwo{N+3}, \dots, \gTwo{2N}
\end{align*}

where $\alpha$ is a uniform nonzero scalar unknown to the adversary. Note that the $(N+1)$ power is omitted.



\subsection*{Consistency Check}
\newcommand{\hOne}[1]{\overline{\gOne{#1}}}
\newcommand{\hTwo}[1]{\overline{\gTwo{#1}}}

Given an alleged set of public parameters $
\left\{ \overline{g_1{^\alpha}}, \dots; \overline{g_2^{\alpha}}, \dots \right \}
$, one can check that it is of the correct form. This is described in a separate document.

\subsection*{Parameter Generation MPC}

\subsubsection*{Notation}
\newcommand{\powers}{\mathrm{Powers}}
For convenience, let $\powers(g_1^\alpha)$ be shorthand for
\[
g_1^\alpha, \gOne{2}, \dots, \gOne{N},  \gOne{N+2}, \gOne{N+3}, \dots, \gOne{2N}, g_2^{\alpha}, \gTwo{2}, \dots, \gTwo{N}, \gTwo{N+2}, \gTwo{N+3}, \dots, \gTwo{2N}
\]

and let $\powers(g_1^\alpha)^\beta$ denote $\powers(g_1^{\alpha\beta})$.
Note that $\powers(g_1^\alpha)^\beta$ can be computed from $\powers(g_1^\alpha)$ and $\beta$ without knowing $\alpha$ by raising each element of $\powers(g_1^\alpha)$ to the corresponding power of $\beta$. 

%\begin{align*}
%&\left(g_1^\alpha\right)^{\beta}, \left(\gOne{2}\right)^{\beta^2}, \dots, \left(\gOne{N}\right)^{\beta^N} \\
%&\left(\gOne{N+2}\right)^{\beta^{N+2}}, \left(\gOne{N+3}\right)^{\beta^{N+3}}, \dots, \left(\gOne{2N}\right)^{\beta^{2N}} \\
%&\left(g_2^{\alpha}\right)^{\beta}, \left(\gTwo{2}\right)^{\beta^2}, \dots, \left(\gTwo{N}\right)^{\beta^N} 
%\end{align*}
%i.e., raising each element of $\powers(g_1^\alpha)$ to the corresponding power of beta.
%Note that $\powers(g_1^\alpha)^\beta = \powers(g_1^{\alpha \beta})$ and that it can be efficiently computed from $\powers(g_1^\alpha)$ and $\beta$ without knowledge of $\alpha$.

Given a vector of $2N-1$ elements of $\G_1$ and $2N-1$ elements of $\G_2$, the consistency check in the previous section allows verifying that there exists a nonzero $\alpha$ for which these elements are $\powers(g_1^\alpha)$. This consistency check is described in a separate document.

\newcommand{\ZKPoK}{\operatorname{ZKPoK}}
\newcommand{\id}{\mathrm{id}}

Let $\ZKPoK$ be a zero knowledge proof of knowledge of exponent scheme where proofs are bound to an arbitrary string (an identity) such that seeing a simulated proof for identity $I$ does not affect soundness of proofs for any identity other than $I$. Write $\ZKPoK(x, I)$ for a zero knowledge proof of knowledge of exponent of $g_1^x$ bound to identity $I$. 
In the random oracle model, $\ZKPoK(x, I)$ can be a Schnorr signature of message $I$ using private key $x$.

\subsubsection*{Protocol}
This protocol is similar to the ZCash ``Powers of Tau" protocol.

Let there be $k$ parties $P_1, \dots, P_k$, each with distinct identity $\id_1, \dots \id_k$. Let $x_0$ be some public fixed nonzero scalar known in advance (e.g., 2).
The first party picks a random scalar $\delta_1$, broadcasts $\powers(g_1^{x_0})^{\delta_1})$ along with $g_1^{\delta_1}$ and $\ZKPoK(\delta_1; \id_1)$, and erases $\delta_1$.
All parties check the consistency of $\powers(g_1^{x_0 \delta_1})$, check that $e(g_1^{\delta_1}, g_2^{x_0}) = e(g_1^{x_0 \delta_1}, g_2)$, and check the $\ZKPoK$ and abort on failure.

The second party picks a random scalar $\delta_2$, broadcasts $\powers(g_1^{x_0 \delta_1})^{\delta_2}$ along with $g_1^{\delta_2}$ and $\ZKPoK(\delta_2; \id_2)$, and erases $\delta_2$.

All parties check the consistency of $\powers(g_1^{x_0 \delta_1 \delta_2})$, check that $e(g_1^{\delta_2}, g_2^{x_0 \delta_1}) = e(g_1^{x_0 \delta_1 \delta_2}, g_2)$, and check that the $\ZKPoK$ is a valid proof of knowledge of exponent for $g_1^{\delta_2}$.

In general, if party $i-1$ broadcast $\powers(g_1^{x_{i-1}})$, then party $i$ will broadcast $\powers(g_1^{x_{i-1}})^{\delta_i}$ along with $g_1^{\delta_i}$ and $\ZKPoK(\delta_i; \id_i)$. All parties will check the consistency of $\powers(g_1^{x_{i-1}})^{\delta_i}$, check that $e(g_1^{\delta_i}, g_2^{x_{i-1}}) = e(g_1^{x_{i-1} \delta_i}, g_2)$, and check that the $\ZKPoK$ is a valid proof of knowledge of exponent for $g_1^{\delta_i}$.

Finally, after the last party has participated (sending $\powers(g_1^{x_k})$) and all parties have done the corresponding consistency / pairing / PoK checks, a scalar $\beta$ is sampled from a public random beacon and the final output parameters are $\powers(g_1^{x_k})^\beta$.

\subsubsection*{Proof of security}
\newcommand{\Adv}{\mathcal{A}}
\newcommand{\Sim}{\mathrm{Sim}}
\begin{claim}
There exists a simulator $\Sim$ such that for all PPT adversaries $\Adv$ corrupting all but one party, assuming $\Adv$ does not cause an abort,
\[
\left\{ \left( \text{$\Adv$'s view during real execution}, \text{Real output params} \right) \right\}
\approx
\left\{ \alpha \gets \$ \::\: \left( \Sim^{\Adv}(\powers(g_1^\alpha)), \powers(g_1^\alpha) \right) \right\}
\]
\end{claim}
\begin{proof}
Let $P_i$ be the (sole) honest party. The simulator has been given $\powers(g_1^\alpha)$ for some uniformly sampled nonzero $\alpha$ but has not been given $\alpha$. The simulator gets to choose $P_i$'s message and the random beacon value, it gets black-box access to the adversary, and it can (by programming a random oracle or setting a CRS) simulate valid proofs for the $\ZKPoK$ scheme. The goal of the simulator is to choose $P_i$'s message and a random beacon value so as to produce valid transcripts giving $\powers(g_1^\alpha)$ as the output parameters, such that the distribution of simulated transcripts is indistinguishable from that of real transcripts generated by running the protocol with the same adversary and an honest $P_i$ (conditioned on having $\powers(g_1^\alpha)$ as output).

The simulator first does the following:
\begin{itemize}
\item Runs $\Adv.P_{1}$ to get $\overline{\powers(g_1^{x_1})}, \overline{\ZKPoK(\delta_1; \id_1)}, \overline{g_1^{\delta_1}}$.
\item Rewinds to extract $\delta_1$ from the $\ZKPoK$
\item \dots
\item Runs $\Adv.P_{i-1}$ to get $\overline{\powers(g_1^{x_{i-1}})}, \overline{\ZKPoK(\delta_{i-1}; \id_{i-1})}, \overline{g_1^{\delta_{i-1}}}$.
\item Rewinds to extract $\delta_{i-1}$ from the $\ZKPoK$
\end{itemize}

Note that an honest $P_i$ would abort unless all consistency checks pass, so since we assume $\Adv$ does not cause an abort, each $\overline{\powers(g^{x_j})}$ is in fact $\powers(g^{x_j})$ (for some $x_j$). Furthermore an honest $P_i$ would abort if any of the proofs of knowledge of exponent for $\overline{g_1^{\delta_j}}$ fail, so by soundness of the $\ZKPoK$ scheme the extracted $\delta_j$ is the correct exponent in $\overline{g_1^{\delta_j}}$. Finally, since honest $P_i$ would check that $e(g_1^{\delta_j}, g_2^{x_{j-1}}) = e(g_1^{x_j}, g_2)$ for all $1 \le j < i$, we know that $x_{i-1} = x_0 \delta_1 \dots \delta_{i-1}$. In particular, the simulator knows $x_{i-1}$, the exponent in the set of powers sent by $P_{i-1}$.

The simulator now picks a random nonzero scalar $r$.
The simulated $P_i$ now broadcasts 
\[
\powers(g_1^\alpha)^{r x_{i-1}} \,,\, (g_1^\alpha)^r, \mathrm{ZKSim}((g_1^{\alpha})^r; \id_i)
\]
where ZKSim is the ZK simulator for the $\ZKPoK$ scheme.

The simulator now runs the remaining $\Adv.P_{i+1}, \dots, \Adv.P_{k}$ and extracts $\delta_{i+1}, \dots, \delta_{k}$ as before. Note that $P_i$'s simulated $\ZKPoK$ is tied to $\id_i$ and so giving it to the adversary does not impact soundness of the remaining adversary-produced $\ZKPoK$, which are tied to $\id_j \ne \id_i$.

As before, the consistency checks ensure that the last party's message is $\powers(g_1^{\alpha r x_{i-1} \Delta})$, where $\Delta = \delta_{i+1} \dots \delta_{k}$. Finally the simulator programs the random beacon value to be $\hat{\beta} = \frac{1}{r x_{i-1} \Delta}$. Now the final output is $\powers(g_1^{\alpha r x_{i-1} \Delta})^{\hat{\beta}} = \powers(g_1^\alpha)$.

We now compare the real view of the adversary to a view simulated by this simulator. All players' messages before $P_i$ are sampled identically (by running the adversary) in the two views.

In the real view, $P_i$'s message is sampled as follows:
\[
\left\{ \delta_i \gets \$ , \powers(g^{x_{i-1}}) \gets \Adv.P_{i-1} \::\: (\powers(g^{x_{i-1}})^{\delta_i},\, g_1^{\delta_i},\, \ZKPoK(\delta_i; \id_i)) \right\}
\]

In the simulated view, $P_i$'s message is sampled as follows, where $r$ is an already-fixed nonzero scalar:
\[
\left\{ \overline{\powers(g^{x_{i-1}})} \gets \Adv.P_{i-1} \::\: (\powers(g^{\alpha})^{r x_{i-1}},\, g_1^{\alpha r},\, \mathrm{ZKSim}(g_1^{\alpha r}; \id_i)) \right\}
\]

$\alpha$ is uniform, nonzero, and independent of $x_{i-1}$, so in both the real and simulated views, $P_i$'s message is distributed as
\[
\begin{Bmatrix}
\powers(g^u) \text{ for uniform nonzero $u$ independent of $x_{i-1}$}, \\
\text{the unique $g_1^z$ such that $e(g_1^z, g_2^{x_{i-1}}) = e(g_1^{u}, g_2)$}, \\
\text{a ZKPoK of exponent of $g_1^z$ using identity $\id_i$}
\end{Bmatrix}
\]
noting that the real and simulated ZKPoKs are indistinguishable by the zero-knowledge property of the ZKPoK scheme. Note also that the distribution of $P_i$'s message is independent of $r$.

All messages up to and including $P_i$'s are indistinguishable in the two views, so the remaining parties' messages (which are in both views generated by running the adversary on all messages so far) are indistinguishable as well.

Finally, the real random beacon value $\beta$ is a uniform nonzero scalar, and the simulated beacon value is
\[
\hat{\beta} = \frac1{rx_{i-1} \Delta}
\]
where $\Delta$ is extracted from the adversary's ZKPoKs after $P_i$. $r$ is uniform nonzero and independent of $x_{i-1}$ and $\Delta$ (since $P_i$'s message is independent of $r$). Therefore $\hat{\beta}$ is uniform nonzero.

The output parameters in the simulated view are uniquely determined by the messages and the random beacon. Thus the real and simulated views are identical.
\end{proof}

\end{document}
